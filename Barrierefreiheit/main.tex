% ══════════════════════════════════════════════════════════════════════════════
% Digitale Barrierefreiheit – Handreichung ZWEK / HSD
% ══════════════════════════════════════════════════════════════════════════════

\documentclass[a4paper,11pt]{article}

% ── Encoding & Sprache ───────────────────────────────────────────────────────
\usepackage[utf8]{inputenc}
\usepackage[T1]{fontenc}
\usepackage[ngerman]{babel}

% ── Seitenlayout ─────────────────────────────────────────────────────────────
\usepackage[a4paper, top=1.0cm, bottom=2.05cm,
            left=1.10cm, right=1.10cm, includefoot]{geometry}
\setlength{\parindent}{0pt}
\setlength{\parskip}{0pt}

% ── Schrift ───────────────────────────────────────────────────────────────────
\usepackage{helvet}
\renewcommand{\familydefault}{\sfdefault}

% ── Pakete ────────────────────────────────────────────────────────────────────
\usepackage{amssymb}
\usepackage{xcolor}
\usepackage{fontawesome5}
\usepackage{mdframed}
\usepackage{ccicons}
\usepackage{fancyhdr}
\usepackage{enumitem}
\usepackage{microtype}
\usepackage{setspace}
\usepackage{accsupp}

% ── Unicode-Zuordnung (Copy-Paste, Suche, Screenreader) ──────────────────────
\input{glyphtounicode}
\pdfgentounicode=1
\usepackage{cmap}

% ── hyperref immer zuletzt ────────────────────────────────────────────────────
\usepackage{hyperref}

% ── Farben ────────────────────────────────────────────────────────────────────
\definecolor{customred}{RGB}{200,0,30}
\definecolor{lightred}{RGB}{255,235,238}
\definecolor{lightgray}{RGB}{245,245,245}
\definecolor{textdunkel}{RGB}{26,26,26}

% ── Typographie ───────────────────────────────────────────────────────────────
\raggedright
\setstretch{1.08}
\hyphenpenalty=200
\tolerance=400
\emergencystretch=2em

% ── PDF-Metadaten ─────────────────────────────────────────────────────────────
\hypersetup{
  pdftitle           = {Digitale Barrierefreiheit in der Hochschullehre},
  pdfauthor          = {Christian-Maximilian Steier, ZWEK – Hochschule Düsseldorf},
  pdfsubject         = {Handreichung Barrierefreiheit},
  pdfkeywords        = {Barrierefreiheit, Lehrmaterialien, Moodle, WCAG},
  pdflang            = {de-DE},
  pdfdisplaydoctitle = true,
  colorlinks         = true,
  linkcolor          = customred,
  urlcolor           = customred,
}

% Icons aus Bookmarks raushalten
\pdfstringdefDisableCommands{%
  \def\faUniversalAccess{}%
  \def\faLaptopCode{}%
  \def\faVideo{}%
  \def\faClipboardList{}%
  \def\faFile{}%
  \def\faChalkboard{}%
  \def\faImage{}%
  \def\faClipboardCheck{}%
  \def\faCheckCircle{}%
  \def\faArrowRight{}%
  \def\faExclamationTriangle{}%
}

% ── Footer ────────────────────────────────────────────────────────────────────
\pagestyle{fancy}
\fancyhf{}
\renewcommand{\headrulewidth}{0pt}
\setlength{\footskip}{18pt}
\fancyfoot[C]{%
  \ccby{}\quad Erstellt vom \textbf{ZWEK} der Hochschule Düsseldorf
  unter \textbf{CC BY 4.0}.\\
  \textsf{\small Quelle: Kompetenzzentrum digitale Barrierefreiheit.nrw –
  \url{https://barrierefreiheit.dh.nrw/materialien}
  \textbullet\ Stand: 2026}}

% ── Accessible Icon Helper ────────────────────────────────────────────────────
% Nur für dekorative Icons in Fließtext/Titeln (NICHT in \item-Labels)
\newcommand{\AIcon}[2]{%
  \BeginAccSupp{method=pdfstringdef,ActualText={#1}}#2\EndAccSupp%
}

% ── Listen-Symbole ────────────────────────────────────────────────────────────
% Für \item-Labels: AccSupp NICHT verwenden – das versteckt das Icon.
% Stattdessen: Icon sichtbar, kein ActualText nötig (rein dekorativ in Listen).
\newcommand{\ja}  {{\textcolor{textdunkel}{\faCheckCircle}}\,}
\newcommand{\tipp}{{\textcolor{textdunkel}{\faArrowRight}}\,}
\newcommand{\warn}{{\textcolor{customred}{\faExclamationTriangle}}\,}
\newcommand{\chk} {\(\square\)\,}

% ── Box-Header ────────────────────────────────────────────────────────────────
% \quad = 1em Abstand zwischen Icon und Titel
\newcommand{\boxheader}[2]{%
  {\small\bfseries\color{customred}#1\hspace{0.4em}#2}\\[-3pt]%
  {\color{customred}\rule{\linewidth}{0.5pt}}\\[2pt]%
}

% ── Themenbox ─────────────────────────────────────────────────────────────────
\newcommand{\themenbox}[3]{%
  \begin{mdframed}[%
    linecolor=customred, linewidth=0.95pt,
    innerleftmargin=5pt, innerrightmargin=5pt,
    innertopmargin=4pt, innerbottommargin=4pt,
    skipabove=0pt, skipbelow=0pt,
    backgroundcolor=white]
    \boxheader{#1}{#2}
    \begin{itemize}[leftmargin=16pt, labelsep=3pt,
                    itemsep=1.6pt, parsep=0pt, topsep=0pt]
      \small #3
    \end{itemize}
  \end{mdframed}%
}

% ── 3-Spalten-Row ─────────────────────────────────────────────────────────────
\newcommand{\rowthree}[3]{%
  \begin{minipage}[t]{0.325\linewidth}#1\end{minipage}\hfill
  \begin{minipage}[t]{0.325\linewidth}#2\end{minipage}\hfill
  \begin{minipage}[t]{0.325\linewidth}#3\end{minipage}\par%
}

% ─────────────────────────────────────────────────────────────────────────────
\begin{document}
\color{textdunkel}

% ── Worttrennung HIER setzen: babel hat jetzt die ngerman-Muster geladen ──────
% Mindestens 4 Buchstaben vor UND nach dem Trennstrich
\lefthyphenmin=4
\righthyphenmin=4

% ── Titel ─────────────────────────────────────────────────────────────────────
\begin{center}
  {\LARGE\bfseries\textcolor{customred}{Digitale Barrierefreiheit\enspace
    \AIcon{(Universeller Zugang)}{\faUniversalAccess}}}
\end{center}

\vspace{0.08cm}

% ── Intro-Box ─────────────────────────────────────────────────────────────────
\fcolorbox{customred!40}{lightred}{%
  \begin{minipage}{\dimexpr\linewidth-2\fboxsep-2\fboxrule}
    \small\raggedright
    Barrierefreie Materialien helfen nicht nur Studierenden mit Behinderung –
    sie verbessern die Qualität für alle.\\
    \textbf{Grundsatz:} Von Beginn an barrierefrei erstellen ist einfacher als
    nachträglich nachbessern.\\
    \textbf{Minimalstandard:} Struktur \textbullet\ Alt-Texte
    \textbullet\ Untertitel.
  \end{minipage}%
}

\vspace{0.30cm}

% ── Reihe 1 ───────────────────────────────────────────────────────────────────
\rowthree
{%
  \themenbox{\faLaptopCode}{Moodle-Kursraum \& H5P}{%
    \item[\ja]   Kursstruktur über \textbf{Abschnitte} organisieren
    \item[\tipp] Texte als \textbf{Textseite} anlegen (HTML oft besser als PDF)
    \item[\ja]   Dokumente als \textbf{.docx} oder \textbf{.pdf} bereitstellen
    \item[\ja]   \textbf{Alt-Texte} für Bilder (Moodle-Seiten/H5P)
    \item[\ja]   H5P: \textbf{Semantische Überschriften} (Heading\,2, 3)
    \item[\warn] Autoplay/autom. Übergänge \textbf{deaktivieren}
    \item[\tipp] Kurze Nutzungshinweise (Bedienung/Tastatur) voranstellen
  }%
}
{%
  \themenbox{\faVideo}{Videos \& Live/Hybrid}{%
    \item[\ja]   \textbf{Untertitel} aktivieren und nachbearbeiten
    \item[\ja]   Video \textbf{pausierbar}, Lautstärke einstellbar
    \item[\ja]   Wichtige Bildinfos \textbf{gesprochen erklären}
    \item[\tipp] \textbf{Transkript/Notizen} (2–5 Stichpunkte)
    \item[\tipp] Live: \textbf{Mikro nutzen}, Fragen wiederholen,
                 kurz zusammenfassen
  }%
}
{%
  \themenbox{\faClipboardList}{Aufgaben/Tests \& Tools}{%
    \item[\ja]   Aufgaben: \textbf{Kriterien + Dateiformat} klar nennen
    \item[\warn] Zeitlimits nur wenn nötig + Alternative anbieten
    \item[\warn] Drag\,\&\,Drop vermeiden oder Alternative bereitstellen
    \item[\ja]   Feedback: Text statt nur Farbe
                 („richtig/falsch + warum")
    \item[\ja]   Word/PPT: \textbf{Barrierefreiheit prüfen}
    \item[\ja]   PDF: \textbf{PAC} testen, Kontrast-Checker nutzen
  }%
}

\vspace{0.25cm}

% ── Reihe 2 ───────────────────────────────────────────────────────────────────
\rowthree
{%
  \themenbox{\faFile}{Dokumente (Word)}{%
    \item[\ja]   \textbf{Formatvorlagen}: Überschrift 1, 2, 3
    \item[\ja]   \textbf{Listen als Listenformat} (nicht manuell nachbauen)
    \item[\ja]   \textbf{Abstände per Absatz} (keine Leerzeilen)
    \item[\ja]   \textbf{Links sprechend} (nicht „hier klicken")
    \item[\ja]   \textbf{Tabellenkopfzeilen} auszeichnen
    \item[\ja]   \textbf{Dokumentsprache + Titel} als Metadaten setzen
    \item[\tipp] Barrierefreiheit prüfen; \textbf{.docx} zusätzlich zur PDF
  }%
}
{%
  \themenbox{\faChalkboard}{Präsentationen (PowerPoint)}{%
    \item[\ja]   Jede Folie: \textbf{Folientitel} im Platzhalter
    \item[\ja]   Schrift mindestens \textbf{18\,pt}, serifenlos
    \item[\ja]   \textbf{Farbe + Symbol} (nicht Farbe allein)
    \item[\ja]   Kontrast: Text 4{,}5:1 / großer Text 3:1
    \item[\ja]   \textbf{Lesereihenfolge} prüfen
                 (Anordnen → Auswahlbereich)
    \item[\tipp] PDF-Export: \textbf{Dokumentstruktur-Tags} aktivieren
    \item[\warn] Keine automatischen Übergänge/Animationen
  }%
}
{%
  \themenbox{\faImage}{Bilder \& Alternativtexte}{%
    \item[\ja]   Inhaltliche Bilder: \textbf{Alt-Text} eintragen
    \item[\ja]   Dekorative Bilder: als \textbf{„dekorativ"} markieren
    \item[\ja]   Alt-Text: Aussage + Kontext („Diagramm zeigt …")
    \item[\ja]   \textbf{Nicht} „Im Bild ist zu sehen …"
    \item[\ja]   Komplex: kurzer Alt-Text + Erläuterung im Text
    \item[\tipp] KI-Tipp: Vorschlag generieren, dann fachlich prüfen
  }%
}

\vspace{0.25cm}

% ── Sofort-Checkliste ─────────────────────────────────────────────────────────
\begin{mdframed}[%
  linecolor=customred, linewidth=1.20pt,
  backgroundcolor=lightgray,
  innerleftmargin=7pt, innerrightmargin=7pt,
  innertopmargin=4pt, innerbottommargin=5pt,
  skipabove=0pt, skipbelow=0pt]
  {\small\bfseries\color{customred}%
    \faClipboardCheck\quad
    Sofort-Checkliste: 10 Maßnahmen mit dem größten Effekt}\\[2pt]
  \begin{minipage}[t]{0.485\linewidth}
    \begin{itemize}[leftmargin=14pt, itemsep=1.6pt, parsep=0pt,
                    topsep=0pt, label=\chk]
      \small
      \item Überschriften-Formatvorlagen konsequent verwenden
      \item Listen als Listen; keine Leerzeilen für Abstände
      \item Serifenlose Schrift, mind.\,18\,pt in Präsentationen
      \item Links sprechend benennen (nicht „hier klicken")
      \item Alt-Texte für alle inhaltlichen Bilder eintragen
    \end{itemize}
  \end{minipage}%
  \hfill%
  \begin{minipage}[t]{0.485\linewidth}
    \begin{itemize}[leftmargin=14pt, itemsep=1.6pt, parsep=0pt,
                    topsep=0pt, label=\chk]
      \small
      \item Dekorative Bilder als „dekorativ" kennzeichnen
      \item Farbe nie als einzigen Informationsträger nutzen
      \item Untertitel für Videos + Kurz-Transkript/Notizen
      \item PDF-Export mit Dokumentstruktur-Tags (wenn PDF nötig)
      \item Barrierefreiheitsprüfer kurz laufen lassen
    \end{itemize}
  \end{minipage}
\end{mdframed}

\end{document}
